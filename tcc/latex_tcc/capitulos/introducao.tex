\chapter{Introdução}

Uma importante tarefa em processamento de sinais analógicos e digitais é o projeto de filtros. Ainda que nos últimos anos o aumento na capacidade de processamento dos computadores e sistemas embarcados tenha possibilitado o desenvolvimento de filtros digitais cada vez mais sofisticados e eficientes, filtros analógicos continuam tendo enorme importância; de fato, alguns autores (por exemplo, \cite{paarmann}) afirmam que \textit{o mundo moderno... não existiria sem os filtros analógicos}.

Todavia, o projeto de ambos tipos de filtros é uma tarefa complexa: embora filtros de pequena ordem possam ser projetados sem dificuldade por meio de funções de transferência, este projeto torna-se mais trabalhoso à medida em que a ordem desejada aumenta. Assim, tornam-se necessárias ferramentas computacionais que tirem do projetista a necessidade de cálculos manuais e suscetíveis a erros.

Neste trabalho propõe-se uma ferramenta computacional interativa para o projeto de filtros analógicos e digitais, desenvolvida empregando-se \textit{software} livre e aplicando-se algoritmos já descritos na literatura. 

\section{Motivação}
As seguintes razões motivaram a escolha do tema e a escrita deste trabalho:
\begin{itemize}
\item Interesse no desenvolvimento de ferramentas em \textit{software} livre, para redução da dependência em soluções proprietárias - muitas vezes de custo e complexidade altos, ou limitadas e sem possibilidade de melhoria;
\item Interesse em aproximar a teoria (análises e cálculos teóricos vistos em sala de aula ou em livros) da prática, reduzir ou eliminar os cálculos e permitir que o usuário dedique-se ao entendimento de conceitos.
\end{itemize}

\section{Objetivos}

O objetivo desse trabalho é o desenvolvimento de uma ferramenta computacional gratuita e de código aberto para a síntese de filtros analógicos e digitais, capaz de realizar o projeto destes conforme descrito na literatura, tanto para fins didáticos quanto para aplicações profissionais, sem estar atrelada a um fabricante específico. Para isso, serão seguidos os passos:

\begin{itemize}
\item Fazer uma breve revisão teórica de conhecimentos sobre filtros;
\item Descrever a ferramenta desenvolvida, junto com as metodologias de desenvolvimento empregadas;
\item Avaliar os resultados obtidos;
\item Propor ideias para a continuidade do trabalho.
\end{itemize}

\section{Estrutura do trabalho}

O capítulo 2 irá realizar uma revisão teórica dos conhecimentos empregados nesse trabalho, sendo feitas considerações iniciais aplicáveis a ambos filtros analógicos e digitais e, após, sendo discutidos assuntos referentes a cada tipo de filtro. 

No capítulo 3 será abordado o processo de desenvolvimento, descrevendo-se as ferramentas e metodologias usadas, seguindo-se um capítulo no qual o \textit{software} desenvolvido será demonstrado e discutido. 

Na finalização deste trabalho, serão apresentadas conclusões sobre os resultados obtidos e sugestões para futuras melhorias e continuidade deste trabalho.

\newpage % Fui obrigado a colocar. Senão a formatação não ficava certa (o título do cap 2 não aparecia na mesma página que o começo dele)
