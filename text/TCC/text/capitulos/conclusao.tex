\chapter{Conclusão}

A possibilidade de uma precisão maior na temperatura  da água e um tempo menor de ajuste é um atrativo bastante interessante que agrega um conforto maior ao banho.

A solução apresentada por esse trabalho propõe um regulador de temperaturas para chuveiros elétricos que possibilita o usuário escolher como parâmetro a temperatura em $^\circ$C para a água do seu banho. O regulador foi implementado com o auxílio de um microcontrolador, um sensor de temperatura, componentes eletrônicos e conhecimentos de engenharia que envolvem sistemas de controle e circuitos eletrônicos.

Os resultados apresentados mostraram que o sistema é eficaz e consegue estabilizar a temperatura em uma faixa aceitável de erro e em um tempo menor do que os chuveiros existentes no mercado, fornecendo ao usuário conforto e economia de energia e água.

Uma grande dificuldade que o projeto apresentou foi o tempo de resposta do sensor de temperatura, o que levou a necessidade da implementação de uma estratégia que visasse a eliminar esse tempo morto da malha de controle para que o controlador não fosse afetado o que poderia causar uma não convergência da estabilidade do sistema. 

Os tópicos a seguir são sugeridos para trabalhos futuros:
\begin{itemize}
\item Testes do sistema de controle com outros sensores de temperaturas mais precisos;
\item Aplicação do sistema de controle utilizando outros métodos para eliminação do tempo morto.
\item Otimização dos coeficientes do controlador para diminuir o tempo até o regime permanente.

\end{itemize}