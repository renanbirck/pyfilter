\chapter{Fundamentação Teórica}

\section{Filtros Analógicos}

\subsection{Famílias de filtros analógicos}

\subsubsection{Filtro de Butterworth}

\subsubsection{Filtro de Bessel}

\subsubsection{Filtro Elíptico}

\subsubsection{Filtro Chebyshev, tipo 1}

\subsubsection{Filtro Chebyshev, tipo 2}

\subsection{Implementação de Filtros}

\subsubsection{Topologia \textit{Sallen-Key}}
\subsubsection{Topologia \textit{Multiple Feedback}}
\subsubsection{Topologia \textit{KHN}}

\section{Filtros Digitais}
\subsection{Famílias de filtros digitais}

\subsection{Implementação de Filtros}

\section{Síntese de Filtros}

\section{Ferramentas de desenvolvimento}
\subsection{Python}

Python é uma linguagem de programação de alto nível, de propósito geral, cuja sintaxe permite ao programador expressar ideias complexas em poucas linhas de código. Atualmente, vários softwares proprietários e de código aberto são desenvolvidos nessa linguagem ou a empregam como linguagem de \textit{script}.

Python foi escolhida para o presente trabalho pelo fato de sua simplicidade, aliada à grande disponibilidade de bibliotecas, facilitar o desenvolvimento.

Para o uso dela em aplicações científicas e de engenharia, existe um conjunto de bibliotecas básicas que adicionam funcionalidade similar a de ferramentas como o MATLAB, permitindo o trabalho com números complexos, matrizes, a geração de gráficos e o cálculo de funções especiais, entre outras funcionalidades necessárias.

\subsection{NumPy}
\subsection{SciPy}
\subsection{matplotlib}
\subsection{PyQt}
