\chapter{Introdução}

Com a evolução da tecnologia da informação, a automação residencial vem deixando de ser um 
assunto de filmes de ficção científica e se tornando parte de nossas vidas. Ela é capaz de 
nos  fornecer  comodidade,  praticidade,  qualidade  de  vida  e  muitos  outros  atributos. 
Intensidade das luzes de uma sala, temperatura de ambientes, acendimento de um forno, 
hoje  em  dia,  podem  ser  controlados  por  um  simples  comando  em  um  \textit{notebook}, 
\textit{Smartphone} ou um \textit{tablet}, trabalhando em conjunto de um \textit{hardware} e um \textit{software} adequados.
Um  aparelho  imprescindível em qualquer residência , e que  poderia ter um foco 
maior de pesquisas na área de automação, é o chuveiro elétrico. Ele funciona
de  forma  bem  simples:  A  água  ao  passar  pela  resistência,  localizada  na  parte  interna  do 
chuveiro, aquece-se devido à alta temperatura da resistência, ocasionada pela passagem de 
corrente elétrica (efeito joule).
Um dos problemas encontrados nesse sistema de chuveiro elétrico convencional 
é o tempo que se leva para ajustar a temperatura da água de acordo com nosso gosto. 
Normalmente há a necessidade de se regular a vazão da água e a posição da estação do chuveiro (geralmente verão, inverno e desligado). Esse ajuste de temperatura leva em torno de dois minutos, o que gera, além de certo desconforto, desperdício de recursos ambientais como água e energia.

Visando a aumentar o nível de conforto e reduzir o tempo gasto para o ajuste, o presente trabalho tem como proposta a implementação de um regulador de temperatura para chuveiros elétricos, onde o usuário irá entrar com a temperatura desejada através de uma interface com botões e displays e o microcontrolador irá realizar o controle do sistema para que a temperatura da água atinja o valor desejável.

O chuveiro utilizado no trabalho é um ThermoSystem modelo 01 que utiliza um sistema de controle de temperatura controlado pela alteração do ângulo de disparo do sinal. Esta alteração é feita pelo usuário ao girar a haste localizada na parte inferior do chuveiro.

O sistema convencional será substituído  por um sistema microcontrolado para que a temperatura final seja igual, ou muito próxima à temperatura escolhida.

    Um sensor de temperatura será utilizado para verificar se a temperatura da água condiz com a temperatura desejada e caso não sejam iguais será calculado o erro entre elas. Este erro servirá de parâmetro de entrada para o controlador Proporcional Integral.
    
    O parâmetro de saída do controlador fornecerá um dado  do qual possibilitará o cálculo de tempo de disparo do TRIAC. Este tempo será calculado inúmeras vezes até que o sistema atinja a estabilidade.
    
\section{Motivação}
As motivações mais importantes para a realização deste trabalho foram as seguintes:
\begin{itemize}
\item Aplicação prática de assuntos abordados durante o curso de Engenharia de Computação como microcontroladores, sistemas de controle, circuitos eletrônicos;
\item Economia e otimização do uso de recursos ambientais, como água e energia elétrica;
\item Automatização de um componente comum a praticamente todas as residências;
\end{itemize}

\section{Objetivos}

Os objetivos principais deste trabalho são:

\begin{itemize}
\item Diminuição do tempo de ajuste de temperatura da água do banho;
\item Economia de água e energia;
\item Conforto na escolha da temperatura precisa em graus Celsius;
\end{itemize}


\section{Estrutura do trabalho}

Esta seção irá comentar sobre a estrutura dos capítulos deste trabalho.

    O capítulo 2 irá apresentar informações sobre os assuntos que foram estudados e pesquisados em diversas fontes e os quais ofereceram conhecimento para que a implementação do projeto fosse realizada.
    
    O capítulo 3 irá tratar dos procedimentos efetuados para o desenvolvimento do regulador.
    
    No capítulo 4 serão mostrados as etapas que foram realizadas para que a montagem do projeto se tornasse possível.
    
    O capítulo 5 irá apresentar os resultados obtidos, através de gráficos construídos pelo \textit{software} matemático Matlab.

    

