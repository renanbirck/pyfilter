\chapter{Desenvolvimento}

Esse capítulo irá discorrer sobre as metodologias e procedimentos utilizados no desenvolvimento da ferramenta.

\section{Estrutura do programa}
\section{Metodologia de desenvolvimento}
\subsection{TDD (Test Driven Development)}

TDD (\textit{Test Driven Development}, na literatura em português encontrado como \textit{Desenvolvimento orientado para testes}) é uma metodologia de desenvolvimento de software que blá blá blá.

Considerou-se essa metodologia adequada para o presente trabalho pois ela fornece uma maneira rápida e eficiente de testar os algoritmos e estruturas de dados, além de obrigar o desenvolvedor a pensar em termos de atender requisitos.

Para adicionar-se uma nova funcionalidade, inicialmente escreve-se e executa-se um teste. Se ele executar sem erros, considera-se o desenvolvimento desta funcionalidade completo; caso contrário, o código é desenvolvido e modificado em pequenos passos até que todos os testes passem com sucesso.

Dessa forma, o desenvolvedor tem maior confiança para realizar alterações em um código: se a mudança provocar falha em algum dos testes, sabe-se exatamente o que causou o defeito.

\subsection{Controle de versão}
